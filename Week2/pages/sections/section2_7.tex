\documentclass[class=report, crop=false]{standalone}
\usepackage[subpreambles=true]{standalone}

\begin{document}
    \section{Lister}
    En uordnet listes standard ses på første item og kan laves på følgende måde:
    \begin{tcblisting}{}
        \begin{itemize}
            \item Almindelig.
            \item[--] Bindestreg.
            \item[$-$] Anden form for bindestreg.
            \item[$oo$] Et valgfrit tegn. 
        \end{itemize}
    \end{tcblisting}
    \noindent En ordnet liste har mulighed for forskellige labes og kan laves på følgende måde:
    \begin{tcblisting}{}
        \begin{enumerate}
            \item Første.
            \item Anden.
        \end{enumerate}
        \begin{enumerate}[label=\arabic*)]
            \item Første.
            \item Anden.
        \end{enumerate}
        \begin{enumerate}[label=(\roman*)]
            \item Første.
            \item Anden.
        \end{enumerate}
        \begin{enumerate}[label=\alph*]
            \item Første.
            \item Anden.
        \end{enumerate}
    \end{tcblisting}
    \noindent Bemærk at der er mulighed for at sætte tegn foran og bagved ex. \textbackslash roman*. Lister kan også være inde i lister og ser således ud:
    \begin{tcblisting}{}
        \begin{enumerate}
            \item Første niveau.
            \item Første niveau.
            \begin{enumerate}
                \item Andet niveau.
                \item Andet niveau.
                \begin{enumerate}
                    \item Tredje niveau.
                    \item Tredje niveau.
                    \begin{enumerate}
                        \item Fjerde niveau.
                        \item Fjerde niveau.
                    \end{enumerate}
                \end{enumerate}
            \end{enumerate}
        \end{enumerate}
    \end{tcblisting}
\end{document}