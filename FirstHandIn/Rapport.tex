\documentclass[10pt]{report}
\usepackage[utf8]{inputenc}
\usepackage[danish]{babel}
\usepackage [T1]{fontenc}
\usepackage[margin=2.5cm, headheight=36pt]{geometry}
\usepackage[hidelinks]{hyperref}
\usepackage{graphicx}
\graphicspath{{figures/}{Billeder/}}
\usepackage{listings}
\usepackage{color}
\usepackage{adjustbox}
\usepackage{tocloft}
\usepackage{listings}
\usepackage{enumitem}
\usepackage{indentfirst}
\usepackage{caption}
\usepackage{float}
\usepackage{lmodern,textcomp}
\usepackage{varwidth}
\usepackage{amsmath}

\usepackage{fourier}
\usepackage{array}
\usepackage{makecell}
\renewcommand\theadalign{bc}
\renewcommand\theadfont{\bfseries}
\renewcommand\theadgape{\Gape[3.5pt]}
\renewcommand\cellgape{\Gape[3.5pt]}


\usepackage{fancyhdr}
\fancypagestyle{fancy}{
  \fancyhf{} % clear all header and footer fields
  \fancyhead[L]{av105 - Andreas Vikke\\mf237 - Martin Frederiksen}
  \fancyhead[R]{\today\\}
  \fancyfoot[R]{Side \thepage}
  \renewcommand{\headrulewidth}{0.4pt}% Line at the header visible
  \renewcommand{\footrulewidth}{0.4pt}% Line at the footer visible
}
\fancypagestyle{plain}{%
  \fancyhf{} % clear all header and footer fields
  \fancyhead[L]{av105 - Andreas Vikke\\mf237 - Martin Frederiksen}
  \fancyhead[R]{\today\\}
  \fancyfoot[R]{Side \thepage}
  \renewcommand{\headrulewidth}{0.4pt}% Line at the header visible
  \renewcommand{\footrulewidth}{0.4pt}% Line at the footer visible
}

\renewcommand\thefootnote{\textcolor{black}{\arabic{footnote}}}

\hypersetup{
colorlinks=true,
linkcolor=black,
citecolor=green,
filecolor=magenta,
urlcolor=cyan
}

\setlength{\parskip}{6pt}

\definecolor{dkgreen}{rgb}{0,0.6,0}
\definecolor{gray}{rgb}{0.5,0.5,0.5}
\definecolor{mauve}{rgb}{0.58,0,0.82}
\definecolor{bluekeywords}{rgb}{0.13,0.13,1}
\definecolor{greencomments}{rgb}{0,0.5,0}
\definecolor{turqusnumbers}{rgb}{0.17,0.57,0.69}
\definecolor{redstrings}{rgb}{0.5,0,0}

\lstset{frame=tb,
  language=Java,
  aboveskip=3mm,
  belowskip=3mm,
  showstringspaces=false,
  columns=flexible,
  basicstyle={\small\ttfamily},
  numbers=none,
  numberstyle=\tiny\color{gray},
  keywordstyle=\color{blue},
  commentstyle=\color{dkgreen},
  stringstyle=\color{mauve},
  breaklines=true,
  breakatwhitespace=true,
  tabsize=3
}

\lstdefinelanguage{FSharp}
                {morekeywords={let, new, match, with, rec, open, module, namespace, type, of, member, and, for, in, do, begin, end, fun, function, try, mutable, if, then, else},
    keywordstyle=\color{bluekeywords},
    sensitive=false,
    morecomment=[l][\color{greencomments}]{///},
    morecomment=[l][\color{greencomments}]{//},
    morecomment=[s][\color{greencomments}]{{(*}{*)}},
    morestring=[b]",
    stringstyle=\color{redstrings}
    }


\date{}

\begin{document}
\begin{titlepage}
  \begin{center}
    \vspace*{1cm}

    \Huge
    \textbf{Homework Assignment}
         
    \vspace{1.5cm}

    \LARGE
    \textbf{Andreas Vikke \& Martin Frederiksen}

    \vfill
  
    \includegraphics[width=0.4\textwidth]{7fe3c3f6-Stego}
    
    \vfill
    
    \Large
    \today\\
    Professionsbachelor i softwareudvikling\\
    Cphbusiness Lyngby\\
    Denmark
         
  \end{center}
\end{titlepage}

\chapter*{1. Description}
\addcontentsline{toc}{chapter}{1.  Description}
\pagestyle{fancy}
\noindent In order to look at self reflection and to judge your assessment of information, you should solve the programming exercise below.

\noindent However - the important thing in this exercise is how you solved it, not the end result.

\noindent At the end of the programming exercise you should have:

\begin{itemize}
  \item A list of \textbf{all} search queries you made to solve it, and timestamps (just copy it from the browser history)
  \item A list all pages you visited to solve it (just copy it from the browser history)
  \item A list of the 3 biggest stumbling blocks you came across and your reflection on why they were problematic (did you misunderstand something, was some of the info you found wrong, did you miss a detail, …)
  \item A brief "every 30 min" diary as explained in the slides (this is more frequent than one would normally do, and is just meant as part of the exercise)
\end{itemize}

\chapter*{2. Løsning}
\addcontentsline{toc}{chapter}{2. Løsning}
\pagestyle{fancy}
\noindent Vi valgte at bruge Python som programmingssprog og vi har bygget det som en docker container. Nedenfor ses en række billeder af hvordan projektet kan køres, samt vores løsnings forslag med tilhørende resultat.

\begin{figure}[H]
  \centering
  \includegraphics[width=\textwidth]{HowToRun.png}
  \caption{Billede af kommandoer der kan bruges til at køre projektet.}
\end{figure}

\begin{figure}[H]
  \centering
  \includegraphics[width=\textwidth]{Dockerfile.png}
  \caption{Billede af vores dockerfil.}
\end{figure}

\begin{figure}[H]
  \centering
  \includegraphics[width=\textwidth]{Pythonkode.png}
  \caption{Billede af vores pythonkode.}
\end{figure}

\begin{figure}[H]
  \centering
  \includegraphics[width=\textwidth]{Result.png}
  \caption{Billede af resultatet.}
\end{figure}

\chapter*{3. Største udfordringer}
\addcontentsline{toc}{chapter}{3. Største udfordringer}
\pagestyle{fancy}
\begin{itemize}
  \item \textbf{Forståelse af opgaven:} det største problem for vores gruppe var at forstå opgaven. Vi prøvede hver især først at decode billedet uden noget meningsfuldt resultat. Vi brugte dernæst lidt tid på lige at få snakket opgaven ordenligt igennem sådan at der ikke var nogen tvivl om hvad der præsist skulle ske.
  \item \textbf{Least significant bit of the blue values:} dette brugte vi lidt tid på at forstå ordenligt i forhold til at vi skal tage det sidste bit i den blå del af pixlen. Hvis vi printede en pixel ud fik vi RGB værdien som (R, G, B, ??) hvor ?? er ??.
  \item \textbf{Little-endian:} her var vi langsomme om at opfatte at ??
\end{itemize}

\chapter*{4. Dagbog}
\addcontentsline{toc}{chapter}{4. Dagbog}
\pagestyle{fancy}

\noindent\textbf{05/01-21 kl. 14.07:}

\noindent\textbf{05/01-21 kl. 14.45:}

\noindent\textbf{05/01-21 kl. 15.20:}


\chapter*{5. Søgning og links}
\addcontentsline{toc}{chapter}{5. Search and pages}
\pagestyle{fancy}
\noindent Acceptance Testing. (s.d.). \textit{Softwaretesting Fundamentals}. Lokaliseret den 29.december 2020 på:\\
\href{https://softwaretestingfundamentals.com/acceptance-testing}{https://softwaretestingfundamentals.com/acceptance-testing}


\end{document}

