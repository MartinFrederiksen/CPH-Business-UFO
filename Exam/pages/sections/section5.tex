\documentclass[class=report, crop=false]{standalone}
\usepackage[subpreambles=true]{standalone}

\begin{document}
    \section{Resultatet}
    Efter kørslen af programmerne kunne vi sammenligne resultaterne, som kan ses herunder. Det ses at C\# programmet kører betydeligt hurtigere end Python programmet.\\
    Resultatets middelværdi lå meget højere i Python, som vi havde forventet, men udover dette er differencen også en del højere, som vi ikke havde forventet. Ved at skrive alle gennemsnitsværdierne ud i stedet for middelværdien, kunne vi plotte de to programmer, som kan ses på Figur \ref{fig:SpeedPlot}. Dette gav et tydeligt indblik i stabiliteten af de to sprog, som viser at Python kører meget mere ustabilt også.
    Da Python er meget langsom om at analysere, fortolke og eksekvere har det ikke være muligt for os at have et array i Python på mere end størrelse 10. Det kunne have været sjovt at kigge på et array på størrelse 100.000 eller 1.000.000 for at give en større præcision af resultatet.
    \begin{tcolorbox}
        \text{C\#: } \[ 311,2 \text{ ns} \pm 13,1  \]
    \end{tcolorbox}
    \begin{tcolorbox}
        \text{Python: } \[ 3312,1 \text{ ns} \pm 88,9  \]
    \end{tcolorbox}

    \begin{tcolorbox}
        \begin{figure}[H]
            \centering
            \includegraphics[width=\textwidth]{SpeedPlot.png}
            \caption{Plot af hastighed på Insertion Sort}
            \label{fig:SpeedPlot}
        \end{figure}
    \end{tcolorbox}

\end{document}