\documentclass[class=report, crop=false]{standalone}
\usepackage[subpreambles=true]{standalone}

\begin{document}
    \section{Matematik}
    Hvis vi skal bruge ligninger kan vi skrive den på samme linje som teksten vi skriver eller skrive dem på sin egen separate linje.
    \begin{tcblisting}{}
        Dette er en ligning skrevet på samme linje som tekst \(\sqrt[3]{8} = 2\). Hvor dette er en ligning der er skrevet på en separat linje: \[ \sqrt{a} = a^\frac{1}{2}, a > 0 \]
    \end{tcblisting}
    \noindent Vi kan beskrive en sum således:
    \begin{tcblisting}{}
        \[ \sum_{n=1}^{\infty} 2^{-n} = 1 \]
    \end{tcblisting}
    \noindent Vi kan beskrive et produkt således:
    \begin{tcblisting}{}
        \[ \prod_{n=1}^{10} n^{2} \]
    \end{tcblisting}
\end{document}