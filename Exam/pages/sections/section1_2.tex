\documentclass[class=report, crop=false]{standalone}
\usepackage[subpreambles=true]{standalone}
\usepackage{enumitem}
\usepackage{csquotes}

\begin{document}

    \section{Hypotese for hastighed af de to sprog.}
        Python og C\# er bygget på to forskellige typer programmeringssprog, som har stor indflydelse på hastigheden af programmet man laver. C\# er et compiled sprog hvilket vil sige at koden er oversat til maskinkode ved brugen af en compiler. På denne måde vil man få et meget effektivt program, som kan køres flere gange. Python er i modsætning til C\# et interpreted programmeringssprog, som skal analysere, fortolke og eksekvere for hver gang programmet bliver kørt.
        \begin{displayquote}
            ”One of the jobs of a designer is to weigh the strengths and weaknesses of each language and then decide which part of an application is best served by a particular language” --- IBM Corporation\cite{ibmcomvsint}
        \end{displayquote}
        Som IBM nævner i \cite{ibmcomvsint} er det en vigtig kvalitet at kunne se styrker og svagheder for hver type sprog, og ud fra det vælge hvilket sprog der er bedst egnet til den type applikation der skal laves. Med disse ting i mente forudser vi at vores InsertionSort algoritme, vil kunne køre hurtigere i C\# end Python. Grundet at Python vil skulle bruge længere tid på at eksekvere koden end C\#, da den skal fortolke koden linje for linje.

\end{document}