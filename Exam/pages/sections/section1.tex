\documentclass[class=report, crop=false]{standalone}
\usepackage[subpreambles=true]{standalone}
\definecolor{ao(english)}{rgb}{0.0, 0.5, 0.0}

\begin{document}
    \section{C\# vs Python}
        Vores gruppe har valgt at undersøge hastighedsforskellen imellem Python og C\#. Denne undersøgelse er blevet foretaget ud fra et stykke kode, vi skrev i starten af faget \textit{Diskret matematik og algoritmer}.
        Kodestumpen blev oprindeligt skrevet i C\#, men i forbindelse med denne opgave omskrev vi hurtigt kodestumpen til Python, så begge stykker kode udfører samme opgave.
        Begge programmeringssprog har sine fordele og ulemper og det er vigtigt at have dem in mente, når der skal vælges er programmeringssprog til en given opgave.
        Som eSparkbiz Technologies skriver i \cite{esparkinfo} har Python og C\# følgende fordele og ulemper:
        \begin{multicols}{2}
        \raggedcolumns
        \begin{tcolorbox}
        \noindent\textcolor{ao(english)}{Python fordele:}
        \begin{itemize}
            \setlength{\itemsep}{0pt}
            \setlength{\parskip}{0pt}
            \item Sproget er portable, hvilket betyder at det kan køress på flere platforme som fx. Linux, Macintosh, windows osv.
            \item Sproget tillader programmøre at udvikle en grafisk brugergrænseflade.
            \item Sproget blev etableret for 28 år siden og derved huser sproget et stort fælleskab.
            \item Sproget er effektivt eftersom der kun skal skrives få linjer kode.
        \end{itemize}
        \end{tcolorbox}
        \columnbreak

        \begin{tcolorbox}
        \noindent\textcolor{ao(english)}{C\# fordele:}
        \begin{itemize}
            \setlength{\itemsep}{0pt}
            \setlength{\parskip}{0pt}
            \item Sproget er skalerbart og nemt at opdatere.
            \item Sproget har hurtig kompilering.
            \item Sproget er type stærkt da det ikke tillader risikable casts.
            \item Sproget er et struktureret programmeringssprog.
            \item Sproget besider krydsplatform evner.
            \item Sproget har et standard bibliotek.
        \end{itemize}
        \end{tcolorbox}
        \end{multicols}

        \begin{multicols}{2}
        \raggedcolumns
        \begin{tcolorbox}
        \noindent\textcolor{red}{Python ulemper:}
        \begin{itemize}
            \setlength{\itemsep}{0pt}
            \setlength{\parskip}{0pt}
            \item Sproget er langsomt.
            \item Sproget kan ikke bruges til mobil computing.
            \item Sprogets database er primitiv og underudviklet.
            \item Sproget har fejl og mangler som let kommer til udtryk.
            \item Sproget behøver meget testing.
        \end{itemize}
        \vspace*{4pt}
        \end{tcolorbox}
        \columnbreak

        \begin{tcolorbox}
        \noindent\textcolor{red}{C\# ulemper:}
        \begin{itemize}
            \setlength{\itemsep}{0pt}
            \setlength{\parskip}{0pt}
            \item Sproget er ikke ment til underordnet programmering.
            \item Sproget har ikke nogen ubegrænset kompiler.
            \item Sproget kræver en kompilering hver gang der er sket en ændring i koden.
            \item Sproget kræver grundig testing.
            \item Sprogets eksekverbare filer kan kun eksekveres på windows.
        \end{itemize}
        \end{tcolorbox}
        \end{multicols}
        \noindent Disse fordele og ulemper er med til at danne et hurtigt overblik over, hvad vi kan forvente af de to sprog. Desuden præger disse fordele og ulemper også vores hypotese.

\end{document}