\documentclass[12pt]{article}
\usepackage[subpreambles=true]{standalone}
\usepackage[utf8]{inputenc}
\usepackage[danish]{babel}
\usepackage[margin=2.5cm, headheight=36pt]{geometry}
\usepackage{graphicx}
\graphicspath{{figures/}{Billeder/}}
\usepackage{subcaption}
\usepackage[pdfpagelabels]{hyperref}
\usepackage[nottoc]{tocbibind}
\usepackage{multicol}
\usepackage{todonotes}
\usepackage{float}

\usepackage{hyperref}
\Urlmuskip=0mu  plus 10mu
\hypersetup{
    colorlinks=true,
    linkcolor=black,
    filecolor=magenta,      
    urlcolor=cyan,
}


\usepackage[most]{tcolorbox}
\usepackage{lipsum}


\usepackage{listings}
\usepackage{xcolor}
\lstloadlanguages{C,C++,csh,Java}
\definecolor{codegreen}{rgb}{0,0.6,0}
\definecolor{codegray}{rgb}{0.5,0.5,0.5}
\definecolor{codepurple}{rgb}{0.58,0,0.82}

\lstdefinestyle{codestyle}{   
    commentstyle=\color{codegreen},
    keywordstyle=\color{magenta},
    numberstyle=\tiny\color{codegray},
    stringstyle=\color{codepurple},
    basicstyle=\ttfamily\footnotesize,
    breakatwhitespace=false,         
    breaklines=true,                 
    captionpos=b,                    
    keepspaces=true,                 
    numbers=left,                    
    numbersep=5pt,                  
    showspaces=false,                
    showstringspaces=false,
    showtabs=false,                  
    tabsize=2
}

\usepackage{import}

\title{Exploration and Presentation - Exam Assignment part 2}
\author{
    Martin, Frederiksen\\
    \texttt{cph-mf237@cphbusiness.dk}
  }
\date{\today}

\begin{document}
\hypersetup{pageanchor=false}
\maketitle

\noindent Jeg har valgt at læse artiklen CAP Theorem.

\noindent Link1: \url{https://www.imaginarycloud.com/blog/sql-vs-nosql/}

\noindent Dette link har jeg valgt fordi det forklare godt, hvordan man burde se sql vs nosql. I har lidt valgt at tage nosql som en helhed og det kan man ikke rigtig. Der er rigtig mange databaser når vi snakker nosql og en del af databaserne er forskellig på hver deres måde, så derfor er det svært at snakke omn nosql som helhed.
\noindent Jeg vil bruge dette til at give læseren en forståelse af at det altså ikke er muligt at snakke om nosql i helhed og på den måde vil i også komme frem til et mere korrekt resultat.

\newline

\noindent Link2: \url{https://www.xplenty.com/blog/the-sql-vs-nosql-difference/}

\noindent Her kan i læse lidt om de forskellige databaser på relationelle databaser og nosql databaser. Dette kan måske være med til at give et overblik over hvad jeg mener i forhold til at skære alle nosql databaser over en kam.
\noindent Jeg ville bruge dette til at understøtte jeres viden om at nosql ikke skal tages som en helhed og eventuelt gøre brug af et par af de databaser der står listet her. Dette kan måske hjælpe til at præcisere jeres resultat.

\end{document}

